\section{\huge{Implementation}}
The code of the exporter can be found in this github repo: https://github.com/eagletrt/eagletrt-telemetria-exporter

\subsection{Frontend}
The frontend is made with Vue.js and resides in the vuejs folder.

\subsubsection{Vue cli configuration}
The webapp is single-page, so vue-router was not added.
We used vuex (https://vuex.vuejs.org/) as store for the data that was external from the components flow.
We used the default linter to a correct and pretty code.

\subsubsection{External modules}
A part from the Vue.js ones, other external modules were imported. 
For the general layout it was used bootstrap-vue (https://www.npmjs.com/package/bootstrap-vue).
For the beautiful scrollbar it was used vue2-perfect-scrollbar (https://www.npmjs.com/package/vue2-perfect-scrollbar).
For the http requests it was used axios (https://www.npmjs.com/package/axios).

\subsubsection{Configuration}
The frontend has a file /src/config.js that contains a simple configuration object that specifies the url for the API calls.

\subsubsection{Assets}
In the /public folder, only a favicon and a logo were added as assets. The logo is displayed on error messages or 
at loading time.
The logo is available both in png and webp, so that it can be downloaded by the browser in a short amount of time.

\subsubsection{Structure}
The /src folder contains only the usual main.js and App.vue, the config.js, the store.js for vuex, that contains 
the biggest logic/controlling part, and two other folders. 
/components is the usual folder containing the vue components
/services contains simple javascript files containing utils and useful functions, such as the api calls and 
the function to download the zip file.

\subsection{Backend}
The frontend is a NodeJS server

\subsubsection{External modules}
There are lots of external modules used by the backend.
express as framework, body-parser to parse the post requests bodies, chalk for coloured logs, compression to compress the served frontend, 
cors for cross-origin requests, helmet for security, morgan for the requests logs, rimraf to remove the temp files and folders, zip-lib to
zip folders, dree for directory-tree inspections and mongodb to query the database.

\subsubsection{Created modules}
Some parts of the backend code were quite articulated and considered reusable for other projects. 
Hence, we decided to publish three npm modules and importing them in the exporter.
The first is mongo-scanner (https://www.npmjs.com/package/mongo-scanner), to retrieve the structure of a mongodb database, such as the
databases and the collections that it contains.
The second is mongoback (https://www.npmjs.com/package/mongoback), to an easily and fully configurable export of data 
using mongoexport under the hoods. In the exporter it is used to export the data as json.
The third is eagletrt-csv (https://www.npmjs.com/package/eagletrt-csv), which is a sort of analogue of mongoback, but it parses 
data supposing it is structured as our database stucture and in the exporter it is useed to export the data as csv.

\subsubsection{Structure}
The /main.js file is the root of the server, where express is imported and the server starts waiting for requests.
the /config.js file contains a json object used as configuration of the server. It contains the server port, 
the mongodb urls, the location of the frontend to serve and the logger colours.
The /frontend folder contains the built vue frontend, which is served statically.
The /utils folder contains js files that provides useful function used in various files.
The /routes folder contains the routes of the API. It contains a routes folder with .route.js files that provide the various routes and an
index.js file that, by using dree, finds all the .route.js files in the routes folder and adds it to the express router.
