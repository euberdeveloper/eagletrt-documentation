\section{\huge{Versions}}
There are two versions of the telemetry exporter. 
This is because after the first version, the way the mongodb database was structured changed and the exporter needed to be updated 
in order to work again.

\subsection{What is a session}
Every time there was a test or race, multiple "sessions" were saved. A session is a period of time were the car is driven by the same pilot,
doing the same type of test/race, in the same circuit, stopping only for short amounts of time.

\subsection{1.0}
Initially, before changing structure, the mongodb was organized in different databases, such as one for each car (chimera and fenice). 
Each collection in a database corresponded to a session. The name of a collection was decided dinamically by the telemetry and consisted
in the name of the pilot, the type of race and the time when the session started. Inside the collection there were only the data-documents
of the session it corresponded. 
Version 1.0 of the exporter had the same ui of Version 2.0 and the only difference was that it reflected the old database structure. The
first column showed the databases, the second the collections of the selected database and the third the selected collections.
To code of the first version is still available on Github, in the branch 1.0 (https://github.com/eagletrt/eagletrt-telemetria-exporter/tree/1.0).

\subsection{2.0}
At some point the structure of the mongodb database changed and we needed to devlope version 2.0 of the exporter. The current mongodb database
structure has only a fixed database "ealetrt". It contains various collections with arbitrary names. The collections have the same functions
that the databases of the old structure had. They are currently four (chimera, fenice, chimera\_test, fenice\_test). Each collection contains 
two type of documents. The session-documents describe a session with pilot name, type of race, timestamp and a key. The data-documents are the
same of the documents in the old structure, with the difference that now they contains a reference to the key of their session document. So,
now, a collection corresponds no more to a single session, but can contain more than one.
Version 2.0 has as first column the list of the collections, as second column the sessions that the selected collection contains and as third 
column, the selected sessions.