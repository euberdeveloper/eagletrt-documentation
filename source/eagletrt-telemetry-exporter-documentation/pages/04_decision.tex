\section{\huge{Decision}}
The exporter is now a webapp served by the raspberry. 
The memebers of the team can simply connect to the same net of the raspberry an use the exporter in any browser.

\subsection{Why Vue.js?}
We decided to implement the fronted using Vue.js, a javascript frontend framework. 
The frontend is a web frontend, because it is very easy and quick to develope and 
because it automatically grants support for most devices. 
The exporter is thought for desktop devices, we decided to avoid making it responsive because we did not need it. 
It is still usable by phones and with not many changes could be made responsive for all devices.  
Using vanilla js would have been too time-consuming and we did not need better performances. 
We thought that Angular was too over-engeenered for small projects like this and among us no one knew React.
Hence, Vue.js was chosen as frontend framework because it is the most known among the members of the team.

\subsection{Why javascript?}
Initially I wanted to use Typescript, a language that transpiles to Javascript and allows type-checking.
Type-checking, expecially with an editor that includes linters, can spare a lot of debugging time and makes 
the code more readable.
Unfortunately, I was the only one that knew Typescript, so we decided to switch to Javascript because it was known by everyone.
This makes the exporter more mantainable if other members of the team will have to deal with its code.

\subsection{Why hosting it in the raspberry?}
We could have made something that ran on the user's computer and that connected to the raspberry. 
The problem was that anyone should have had it in its own computer and furthermore they would have needed to 
install NodeJS (also mongodb-utils for export as json).
We made the exporter as a normal webapp because now the user could be anyone near the telemetry with almost any device.
The exporter impact on the telemetry performance is negligible, considering also the fact that it mostly used when 
the car is not running.
The code does not change frequently and connecting via ssh to the raspberry in order to update it is not annoying.

\subsection{Why NodeJS?}
NodeJS is one of the most cutting-edge envinronment to develope server like this. 
It was already in the telemetry for other purposes, so we did not have to install it only for the exporter.
The previous command-line version was already written in NodeJS and we kept big parts of that code for the backend.
We used NodeJS also because a main.js file with a quite short amount of code is enough to serve efficiently a frontend.
The backend has the purpose of serving the frontend and providing it the API to export the data.

\subsection{Why json and csv?}
The exporter allows the data to be downloaded as json or csv. These are the most flexible and useful formats for the team needs.
The json is mostly used to pass the collections from the raspberry mongodb to another pc's mongodb, 
this is also why the backend uses internally the mongoexport util. 
The csv is very useful for mathematical elaborations with MatLab or python and is used for example during the tests, 
when we want to check
if the data that we are saving is valid. The csv data is taken with mongodb queries and parsing to csv their result.