\section{\huge{Concept}}
After succeeding in making the telemetry work, we needed a quick way to safely export the gathered data. 
Initially we were used to manually export the data, using ssh and scp. When the number of the collections 
started to increase and we began loosing time to export the data during the tests, we started thinking to 
a better solution.

\subsection{Introduction}
The telemetry is a raspberry that resides in the car, plugged to the canbus. 
It listens to the can and gps messages and saves information on a local MongoDB. 
During the first tests we noticed that we were loosing a big amount of time to export the database data, 
that was done manually with ssh and scp. The purpose of the exporter is to provide a quick and productive way 
to export the data. 

\subsection{Evolution}
The idea of a webapp as an exporter came only after some time. Before thinkind it was necessary, 
others solutions were thought and partially implemented.

\subsubsection{Without exporter}
Exporting the telemetry data manually was very long and annoying. 
The name of the MongoDB collections were made dinamically by the telemetry itself in order 
to provide useful information of their content, such as driver, type of race and timestamp. 
When exporting manually the collections, we needed to connect with ssh to the raspberry. 
After that we needed to list the collections with mongo and choose which were to be exported. 
Then we needed to use mongoexport for each of them, writing manually the collection name and the 
generated json path. Lastly we had to use scp to pass the json files from the raspberry to our pc.

\subsubsection{Using bash}
Initially we thought to write a simple bash script to automatize the job. But soon, before finishing it, 
we noticed that it would not have been enought to simplify the job. We would have stil had to connect via 
ssh to the telemetry.

\subsubsection{Writing a command line program}
After abandoning the idea of a bash script, we wrote a prototype of a command-line exporter. 
It was written with NodeJS (already installed in the raspberry) and allowed to export the collections as 
json (with mongoexport) or as csv. This solution was already quite good and quick: the user could select 
the collections from a list and it was run by the pc without using ssh.

\subsubsection{Writing a webapp}
The command line program was already good, but we thought that it would have been great to enhance the exporter 
providing an intuitive GUI, so that it would habe been easier to everyone to learn how to export the data. 
Furthermore, with the command line exporter the user had to have nodejs and mongodb-utils installed in his pc.