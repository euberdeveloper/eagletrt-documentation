\section{\huge{Decision}}

\subsection{Why nodejs?}
There were no performance reasons to choose a low level language such as C to write this project. We decided to write this
parser during the tests, so we needed to have it quickly. A good choice could have been using python, but I decided to
write it in Javascript because I knew it better. The parser is a simple script that runs over nodejs and takes the ubx
inputs in order to generate the json or the csv parsed outputs.

\subsection{Why a command line program?}
Like with the telemetry exporter, we could have written this parser with a GUI, developing a webapp or something. 
But this project was less important then the exporter and would have been used by only one or two members of the team.
Hence we decided to stay with the simpler solution that was a normal command line program.

\subsection{Why GGA, GGL and RMC messages?}
Only the gps messages GGA, GGL and RMC are parsed, all the other messages are ignored by the parser. Initially only the 
car coordinates were needed, in order to plot the track on Google Earth and get a first raw evidence that they worked properly.
After that, the analisys team needed the coordinates and the speed of the car. The GGA, GGL and RMC messages are the 
ones that together provide these values and the other messages were consequently ignored. Also the message VTG provided
the speed of the car, but we discarded it because it had no timestamp.

\subsection{Why NaN on some values?}
The values that are extracted are latitude longitude timestamp altitude speed and course. The coordinates are converted in a more suitable format.
Not all messages contain every parameter and initially every parameter that is not contained in a message was set to NaN.
Later we decided to change it with NaN, becuase it was easier to handle with MatLab.