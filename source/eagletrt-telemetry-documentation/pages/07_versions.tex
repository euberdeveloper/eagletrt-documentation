\section{\huge{Versions}}
Many changes have been done to the telemetry.

\subsection{Mongodb organisation}

\subsubsection{What is a session}
Every time there was a test or race, multiple "sessions" were saved. A session is a period of time were the car is driven by the same pilot,
doing the same type of test/race, in the same circuit, stopping only for short amounts of time.

\subsubsection{First mongodb organisation}
Initially, before changing structure, the mongodb was organized in different databases, such as one for each car (chimera and fenice). 
Each collection in a database corresponded to a session. The name of a collection was decided dinamically by the telemetry and consisted
in the name of the pilot, the type of race and the time when the session started. Inside the collection there were only the data-documents
of the session it corresponded. 

\subsubsection{Current mongodb organisation}
At some point the structure of the mongodb database changed and we needed to devlope version 2.0 of the exporter. The current mongodb database
structure has only a fixed database "ealetrt". It contains various collections with arbitrary names. The collections have the same functions
that the databases of the old structure had. They are currently four (chimera, fenice, chimera\_test, fenice\_test). Each collection contains 
two type of documents. The session-documents describe a session with pilot name, type of race, timestamp and a key. The data-documents are the
same of the documents in the old structure, with the difference that now they contains a reference to the key of their session document. So,
now, a collection corresponds no more to a single session, but can contain more than one.

\subsection{Code organisation}

\subsubsection{During the first test}
During the first test on the machine, the code pieces (can, mongodb, mqtt, messages parsing) were mostly done. The problem was that the code
was mostly in an unique main.c file and was very difficult to understand and mantain. For a bug that was difficult to find in the main.c
we managed to write a javascript version in a day and used it during the very first telemetry test. (Of course it was nowhere near as performant as c)

\subsubsection{The first reorganisation}
After the first test, we begun to split the c code in more modules. The biggest problem was the c code related to the telemetry data structure.

\subsubsection{The code generator}
We came up with the javascript code generator and used it to generate the c code related to the telemetry data structure

\subsubsection{Another reorganisation}
Before another test we reorganised the code again. We made the state machine as it is now, implemented with a matrix of transition functions (before it was only a switch).
We added also the /utils and the /services folders. 

\subsubsection{Branching the generator}
After seeing the generator was stable and we did not need to modify it frequently, we decided to rewrite it totally in Typescript and in a clearer
code and to publish it as an npm module.

\subsubsection{GPS}
We added the new GPS with the reading of its serial port and the addition of its messages to the structure.

\subsubsection{Threads}
During the last test, we noticed that the addition of the GPS made the telemetry too slow. Only when the car was in run state, the number
of can messages increased significantly and added to the GPS serial port reading overhead it started loose messages and data. We solved this
problem by making two threads: one that read the canbus and another that read the GPS serial port.

