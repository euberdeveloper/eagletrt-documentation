\section{\huge{Concept}}
After managing to build an electric car we needed a telemetry. It was essential to have something that joined all the sensors and gave us an
easy and unique way to understand what happened in the car. As almost every team, we could have bought an already-made solution and plugged it
to the canbus. But that was not the philosophy of ours. We built a telemetry from scratch, projecting and programming it.

\subsection{Requirements}
All the sensors, the steering wheel and the ecu are plugged to the canbus. They send and receive all the messages through that. The simplest 
debugging method could be done is save a can log, task that is actually currently done by the steering wheel by using can-utils (https://github.com/linux-can/can-utils).
But having a raw can log is of course not enough and it is also uncomfortable. So we decided to create a telemetry that read all the messages 
from the canbus and parsed them in a more readable and handable way. After some time, when we added a new base-rover GPS, we decided to plug the
rover GPS directly on the telemetry and read its messages through its serial port.

\subsection{Purpose}
The purpose of the telemetry is making the teams know in a comfortable and easy way what happened in the car. It joins all the messages from the canbus
and the GPS in one place and handles them in a desidered way. It stores the data in a local and easily exportable database and forwards the data 
via mqtt so that some pseudo-real-time applications can be ran. The data can then be managed as wanted, mostly used by the analysis team.