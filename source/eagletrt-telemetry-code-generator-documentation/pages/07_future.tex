\section{\huge{Future}}

\subsection{Storic}
The module was initially a javascript program that ran on nodejs, was located in the same repo of the telemetry and was written in three days.
After that it was debugged by using it and improved, until it was eventually moved in an own repository. The last modifies reguarded the traduction traduction
to Typescript and the complete reorganization of the code. This is because the telemetry C code needed this module and without it would take
lots of time to write manually the code that depends on the data structure. Having only I worked on this project, I had to organize the code and
make it clearer so that any other developer could subsequently maintain it quite easily. A .travis.yml was also added so that the linting and 
documentation site update was done automatically by Travis.ci (https://travis-ci.org).

\subsection{Future}
The module is now stable, but it will be probably extended to automatize other parts of the c code.

\subsubsection{Config interpretation}
The telemetry uses a config.json file to general configuration, such as mqtt and mongodb uri and other stuff. The json config file is quite
articulated is interpreted with JSMN (https://zserge.com/jsmn/) which is very cool but requires lots of code. In the future, this task could
be automatized with this code generator.

\subsubsection{Message parsing}
Another task, more important, that can and probably will be authomatized in the future is the can message parsing. Currently over 85\% of the time
used to maintain and improve the telemetry is occupied by the part that parses the can messages and stores the values in the data structure.
This is because that part of code is, again, repetitive, long and changes frequently. With the new car fenice the messages ids will be generated
automatically and given another json file (messages.json) together with the current structure.json file properly modified, we think that it will
be possible to authomatize also the task of parsing the can messages. This was actually (only conceptually because fenice was at the time still to be
made) already done and the example can be find following this branch of this repo https://github.com/euberdeveloper/telemetria-c/tree/newTask.